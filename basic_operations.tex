
% polynomial expressions

% functions

% expand( ), factor( ), collect( ), together( ), apart( )
%% maybe a table best suits this part.

% simplification: simplify, sqrt denest, fu, trigsimp

\subsection{Simplification}

The generic way to simplify an expression is by calling the \texttt{simplify}
function.
It must be emphasized that simplification is not an unambigously defined
mathematical operation~\cite{Carette2004understanding}.
The \texttt{simplify} function applies several simplification routines along
with some heuristics to make the output expression as ``simple'' as possible.

It is often preferable to apply more directed simplification functions. These
apply very specific rules to the input expression, and are often able to make
guarantees about the output (for instance, the \texttt{factor} function, given
a polynomial with rational coefficients in several variables, is guaranteed to
produce a factorization into irreducible factors).
Table~\ref{simplify-table} lists some common simplification functions.

% TODO: add a simple example for each
% TODO: fix the formatting
\label{simplify-table}
\begin{tabular}{l|p{0.7\linewidth}}
expand & expand the expression \\
factor & factor a polynomial into irreducibles \\
collect & collect polynomial coefficients \\
cancel & rewrite a rational function as $p/q$ with common factors canceled \\
apart & compute the partial fraction decomposition of a rational function \\
trigsimp & simplify trigonometric expressions~\cite{fu2006automated} \\
\end{tabular}

Substitutions are performed through the \texttt{.subs} method, which is
sensible to some mathematical properties while matching, such as
associativity, commutativity, additive and multiplicative inverses, and
matching of powers.

%% TODO: add examples

\subsection{Calculus}

% Calculus (differentiation, integration, limits). Note that algorithm
% descriptions will go in the algorithms section.

Derivatives can be computed with the \verb|diff| function.

\begin{verbatim}
>>> diff(sin(x), x)
cos(x)
\end{verbatim}

Unevaluated \verb|Derivative| objects are also supported.

\begin{verbatim}
>>> expr = Derivative(sin(x), x)
>>> expr
Derivative(sin(x), x)
\end{verbatim}

Unevaluated expressions can be evaluated with the \verb|doit| method.

\begin{verbatim}
>>> expr.doit()
cos(x)
\end{verbatim}

% TODO: A more interesting example here
Integrals can be analogously calculated either with the \verb|integrate|
function, or the unevaluated \verb|Integral| objects.
\begin{verbatim}
>>> integrate(sin(x), x)
-cos(x)
>>> expr = Integral(sin(x), x)
>>> expr
Integral(sin(x), x)
>>> expr.doit()
-cos(x)
\end{verbatim}
Definite integration can be calculated with the same method, by specifying a
range of the integration variable. The following computes $\int_0^1\sin(x)\,dx$.
\begin{verbatim}
>>> integrate(sin(x), (x, 0, 1))
-cos(1) + 1
\end{verbatim}

SymPy implements a combination of the Risch
algorithm~\cite{bronstein2005integration}, table lookups, a reimplementation
of Manuel Bronstein's ``Poor Man's Integrator''~\cite{Bronstein2005pmint}, and
an algorithm for computing integrals based on Meijer G-functions. These allow
SymPy to compute a wide variety of indefinite and definite integrals.
% TODO: What is the best citation for the Meijer G-function algorithm.
% TODO: Add some examples here.

Summations and products are also supported, via the evaluated \verb|summation|
and \verb|product| and unevaluated \verb|Sum| and \verb|Product|, and use the
same syntax as \verb|integrate|.

Summations are computed using a combination of Gosper's algorithm and an
algorithm that uses Meijer G-functions. Products are computed via some
heuristics.
% TODO: Citations for Gosper and Meijer G-function algorithms
% TODO: Are there other summation algorithms implemented?

The limit module implements the Gruntz algorithm~\cite{Gruntz1996limits} for
computing symbolic limits.

The following examples compute $\lim_{x\to 0} \frac{\sin(x)}{x}=1$ and
$\lim_{x\to 0}\left(2 e^{\frac{1 - \cos{\left (x \right )}}{\sin{\left (x \right )}}} -
  1\right)^{\frac{\sinh{\left (x \right )}}{\operatorname{atan}^{2}{\left (x
      \right )}}} = e$, respectively.

\begin{verbatim}
>>> limit(sin(x)/x, x, 0)
1
>>> limit((2*E**((1-cos(x))/sin(x))-1)**(sinh(x)/atan(x)**2), x, 0)
E
\end{verbatim}

\subsection{Printers}

SymPy has a rich collection of expression printers for displaying expressions
to the user. By default, an interactive Python session will render the
\verb|str| form of an expression, which has been used in all the examples in
this paper so far.

\begin{verbatim}
>>> phi0 = Symbol('phi0')
>>> str(Integral(sqrt(phi0), phi0))
Integral(sqrt(phi0 + 1), x)
\end{verbatim}

Expressions can be printed with 2D monospace text with \verb|pprint|. This
uses Unicode characters to render mathematical symbols such as integral signs,
square roots, and parentheses. Greek letters and subscripts in symbol names
are rendered automatically.

Alternately, the \verb|use_unicode=False| flag can be set, which causes the
expression to be printed using only ASCII characters.

\begin{verbatim}
>>> pprint(Integral(sqrt(phi0 + 1), phi0), use_unicode=False)
  /
 |
 |   __________
 | \/ phi0 + 1  d(phi0)
 |
/
\end{verbatim}

The function \verb|latex| returns a \LaTeX{} representation of an expression.

\begin{verbatim}
>>> print(latex(Integral(sqrt(phi0 + 1), phi0)))
\int \sqrt{\phi_{0} + 1}\, d\phi_{0}
\end{verbatim}

Users are encouraged to run the \verb|init_printing| function at the beginning
of interactive sessions, which automatically enables the best pretty printing
supported by their environment. In the Jupyter notebook or
qtconsole~\cite{perez2007ipython} the \LaTeX{} printer is used to render
expressions using MathJax or \LaTeX{} if it is installed on the system. The 2D
text representation is used otherwise.

Other printers such as MathML are also available. SymPy uses an extensible
printer subsystem which allows users to customize the printing for any given
printer, and for custom objects to define their printing behavior for any
printer. SymPy's code generation capabilities, which we will not discuss
in-depth here, use the same printer model.
