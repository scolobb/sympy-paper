SymPy calculates limits using the Gruntz algorithm, as described in%
~\cite{Gruntz1996limits}. The basic idea is as follows: any limit can be
converted to a limit $\lim_{x\to\infty} f(x)$ by substitutions like
$x\to{1\over x}$.
Then we identify in $f(x)$ the most varying subexpression $\omega$ (that
converges to zero as $x\to\infty$ the fastest from all subexpressions), and
expand $f(x)$ into a series with respect to $\omega$. Any positive powers of
$\omega$ converge to zero. If there are negative powers of $\omega$, then the
limit is infinite. The constant term (independent of $\omega$, but could depend
on $x$) then determines the limit (one might need to recursively apply the
Gruntz algorithm on this term to determine the limit).

To determine the most varying subexpression, we first need to define
comparability classes, by calculating $L$:
\begin{equation}
L\equiv \lim_{x\to\infty} {\log |f(x)| \over \log |g(x)|}
\end{equation}
And then we define the $<$, $>$ and $\sim$ operations as follows: $f>g$ when
$L=\pm\infty$ (we say that $f$ is more rapidly varying than $g$, i.e., $f$ goes
to $\infty$ or $0$ faster than $g$, $f$ is greater than any power of $g$),
$f<g$ when $L=0$ ($f$ is less rapidly varying than $g$) and $f\sim g$ when
$L\neq 0,\pm\infty$ (both $f$ and $g$ are bounded from above and below by
suitable integral powers of the other). Here are some examples of comparability
classes:
$$2 < x < e^x < e^{x^2} < e^{e^x}$$
$$2\sim 3\sim -5$$
$$x\sim x^2\sim x^3\sim {1\over x}\sim x^m\sim -x$$
$$e^x\sim e^{-x}\sim e^{2x}\sim e^{x+e^{-x}}$$
$$f(x)\sim{1\over f(x)}$$

Let us now illustrate the Gruntz algorithm on the following example:
\begin{equation}
    \label{gruntz_example_fn}
f(x) = e^{x+2e^{-x}} - e^x + {1\over x} \,.
\end{equation}
We want to calculate $\lim_{x\to\infty} f(x)$.
We first determine the set of most rapidly varying subexpressions, the so
called \textit{mrv set}, for \eqref{gruntz_example_fn} we obtain
$\{e^x, e^{-x}, e^{x+2e^{-x}}\}$. These are all subexpressions of
\eqref{gruntz_example_fn} and they all belong to the same comparability class.
We can do this calculation using SymPy as follows:

\begin{verbatim}
>>> from sympy.series.gruntz import mrv
>>> mrv(exp(x+2*exp(-x))-exp(x) + 1/x, x)[0].keys()
dict_keys([exp(x + 2*exp(-x)), exp(x), exp(-x)])
\end{verbatim}



Next we take any item $\omega$ from mrv that converges to zero for
$x\to\infty$.  We get $\omega=e^{-x}$. If such a term is not present in the mrv
set (i.e. all terms converge to infinity instead of zero), we can use the
relation $f(x)\sim {1\over f(x)}$.

Next step is to rewrite the mrv in terms of $\omega$: $\{{1\over\omega},
\omega, {1\over\omega}e^{2\omega}\}$. Then we substitute for the original
subexpressions back into $f(x)$ and expand with respect to $\omega$:
\begin{equation}
    \label{gruntz_example_fn2}
f(x) = {1\over x}-{1\over\omega}+{1\over\omega}e^{2\omega}
     = 2+{1\over x} + 2\omega + O(\omega^2)
\end{equation}

Since $\omega$ is from the mrv set, then in the limit $x\to\infty$ we have
$\omega\to0$ and so $2\omega + O(\omega^2) \to 0$ in
\eqref{gruntz_example_fn2}:
\begin{equation}
f(x) = {1\over x}-{1\over\omega}+{1\over\omega}e^{2\omega}
    = 2+{1\over x} + 2\omega + O(\omega^2)
    \to 2 + {1\over x}
\end{equation}

Since the result ($2+{1\over x}$) still depends on $x$,
we iterate the above procedure on the result until we get just a number
(independent of $x$), which is the final limit. In our case the limit is $2$,
as can be verified by SymPy:

\begin{verbatim}
>>> limit(exp(x+2*exp(-x))-exp(x) + 1/x, x, oo)
2
\end{verbatim}

In general, when we expand in terms of $\omega$, we obtain:
\begin{equation}
f(x) = \underbrace{O\left({1\over \omega^3}\right)}_\infty
    + \underbrace{C_{-2}(x)\over \omega^2}_\infty
    + \underbrace{C_{-1}(x)\over \omega}_\infty
    + {C_{0}(x)}
    + \underbrace{C_{1}(x)\omega}_0
    + \underbrace{O(\omega^2)}_0
\end{equation}
The positive powers of $\omega$ are zero. If there are any negative powers of
$\omega$, then the result of the limit is infinity, otherwise the limit is
equal to $\lim_{x\to\infty} C_0(x)$. The expression $C_0(x)$ is simpler than
$f(x)$ and so the algorithm always converges. A proof of this, as well as
further details are given in Gruntz's Ph.D. thesis~\cite{Gruntz1996limits}.

