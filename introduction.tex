SymPy is a full featured computer algebra system (CAS) written in the Python
programming language. It is open source, licensed under the extremely
permissive 3-clause BSD license, which allows anyone to reuse SymPy or its source code,
even for commercial purposes. SymPy was started by Ond\v{r}ej \v{C}ert\'{\i}k
in 2005, and it has since grown into a large open source project, with over
500 contributors.
% citation?

SymPy is written entirely in the Python programming language,
% cite Python?
which is also the language used to interact with it, both programmatically and
interactively. Python is a popular dynamically typed programming language that
has a focus on ease of use and readability. It also a very popular language
for scientific computing and data science, with a wide range of useful
libraries.
% Cite numpy, scipy, pandas
% We could also cite
% https://stackoverflow.com/research/developer-survey-2016#most-popular-technologies-per-occupation
Unlike many CASs, SymPy does not invent its own programming language. Python
is used both for the internal implementation and the user interaction.
Exclusively using Python in this way makes it easier for people already
familiar with the language to use SymPy, or to do development on it. It also
lets the SymPy developers focus on mathematics, rather than language design.

SymPy is designed with a strong focus that it be usable as a library. This
means that extensibility is important in its API design. This is also one of
the reasons SymPy makes no attempt to extend the Python language itself. The
goal is for users of SymPy to be able to import SymPy alongside other Python
libraries in their workflow, whether that is an interactive workflow or
programmatic use as part of a larger library.
% cite https://www.python.org/dev/peps/pep-0020/

SymPy does not have a built in graphical user interface (GUI), however, when
used in the Jupyter Notebook
% citation
SymPy expressions will pretty print using MathJax.
% citation
