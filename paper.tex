% SIAM Article Template
\documentclass[review]{siamart0216}

\usepackage{lmodern}
\usepackage[T1]{fontenc}
\usepackage[utf8]{inputenc}

\usepackage{hyperref}
\usepackage{graphicx}
\usepackage{amsmath}
\usepackage{caption}
\graphicspath{ {images/} }

\usepackage{upquote}

\usepackage{amsmath}
\usepackage{url}
\usepackage{hyperref}

% this is required for all the \url{} commands in the bib file
%\usepackage{hyperref}

% for nice units
\usepackage{siunitx}

% for images: png, pdf, etc
\usepackage{graphicx}

% for nice table formatting, i.e., /toprule, /midrule, etc
\usepackage{booktabs}

% to allow for \verb++ declarations in captions.
\usepackage{cprotect}

% to allow usage of \mathbb symbols
\usepackage{amssymb}

\usepackage{longtable}

\title{SymPy: Symbolic Computing in Python}

\input{authors}

\begin{document}
\maketitle

\section{Introduction}

%% What sympy is, where to download etc.
%%
%% List other major CASs.
%%
%% Why SymPy.

SymPy is a full featured computer algebra system (CAS) written in the Python
programming language. It is open source, licensed under the extremely
permissive 3-clause BSD license, which allows anyone to reuse SymPy or its source code,
even for commercial purposes. SymPy was started by Ond\v{r}ej \v{C}ert\'{\i}k
in 2005, and it has since grown into a large open source project, with over
500 contributors.
% citation?

SymPy is written entirely in the Python programming language,
% cite Python?
which is also the language used to interact with it, both programmatically and
interactively. Python is a popular dynamically typed programming language that
has a focus on ease of use and readability. It also a very popular language
for scientific computing and data science, with a wide range of useful
libraries.
% Cite numpy, scipy, pandas
% We could also cite
% https://stackoverflow.com/research/developer-survey-2016#most-popular-technologies-per-occupation
Unlike many CASs, SymPy does not invent its own programming language. Python
is used both for the internal implementation and the user interaction.
Exclusively using Python in this way makes it easier for people already
familiar with the language to use SymPy, or to do development on it. It also
lets the SymPy developers focus on mathematics, rather than language design.

SymPy is designed with a strong focus that it be usable as a library. This
means that extensibility is important in its API design. This is also one of
the reasons SymPy makes no attempt to extend the Python language itself. The
goal is for users of SymPy to be able to import SymPy alongside other Python
libraries in their workflow, whether that is an interactive workflow or
programmatic use as part of a larger library.
% cite https://www.python.org/dev/peps/pep-0020/

SymPy does not have a built in graphical user interface (GUI), however, when
used in the Jupyter Notebook
% citation
SymPy expressions will pretty print using MathJax.
% citation


\section{Architecture}
\label{sec:architecture}


%% I volunteer to write this section. --Aaron
%%
%% Representing symbolic expressions using Python objects

\subsection{The Core}

Every symbolic expression in SymPy is an instance of a Python class.
Expressions are represented by expression trees. The operators is represented
by the type of an expression and the child nodes are stored in the
\texttt{args} attribute. A leaf node in the expression tree has an empty
\texttt{args}.
The \texttt{args} attribute is provided by the class \texttt{Basic},
which is a superclass of all SymPy objects and
provides common methods to all SymPy tree-elements.
For example, take the expression $xy + 2$.

\begin{verbatim}
>>> from sympy import *
>>> x, y = symbols('x y')
>>> expr = x*y + 2
\end{verbatim}

The expression \texttt{expr} is an addition, so it is of type \texttt{Add}. The child
nodes of \texttt{expr} are \texttt{x*y} and \texttt{2}.

\begin{verbatim}
>>> type(expr)
<class 'sympy.core.add.Add'>
>>> expr.args
(2, x*y)
\end{verbatim}

We can dig further into the expression tree to see the full expression. For
example, the first child node, given by \texttt{expr.args[0]} is
\texttt{2}. Its class is \texttt{Integer}, and it has empty \texttt{args},
indicating that it is a leaf node

\begin{verbatim}
>>> expr.args[0]
2
>>> type(expr.args[0])
<class 'sympy.core.numbers.Integer'>
>>> expr.args[0].args
()
\end{verbatim}

The function \texttt{srepr} give a string representing a valid Python code,
containing all the nested class constructor calls to create the given expression:

\begin{verbatim}
>>> srepr(expr)
"Add(Mul(Symbol('x'), Symbol('y')), Integer(2))"
\end{verbatim}

Every SymPy expression satisfies a key invariant, namely,
\texttt{expr.func(*expr.args) == expr}. This means that expressions are
rebuildable from their \texttt{args}\footnote{\texttt{expr.func} is used
  instead of \texttt{type(expr)} to allow the function of an expression to be
  distinct from its actual Python class. In most cases the two are the same.}.
Here, we note that in SymPy, the \texttt{==} operator represents exact
structural equality, not mathematical equality. This allows us to test if any
two expressions are equal to one another as expression trees.

Python allows classes to overload operators. The Python interpreter translates
above \texttt{x*y + 2} to, roughly,
\verb|(x.__mul__(y)).__add__(2)|. \texttt{x} and \texttt{y}, returned from
the \texttt{symbols} function, are \texttt{Symbol} instances. The \texttt{2}
in the expression is processed by Python as a literal, and is stored as
Python's builtin \texttt{int} type. When \texttt{2} is called by the
\verb|__add__|, it is converted to the SymPy type \texttt{Integer(2)}. In
this way, SymPy expressions can be built in the natural way using Python
operators and numeric literals.

One must be careful in one particular instance. Python does not have a builtin
rational literal type. Given a fraction of integers in \texttt{1/2}, Python
will perform floating point division and produce \texttt{0.5}\footnote{This is
  the behavior in Python 3. In Python 2, \texttt{1/2} will perform integer
  division and produce \texttt{0}, unless one uses \texttt{from \_\_future\_\_
    import division}}. Because Python uses eager evaluation, expressions like
\texttt{x + 1/2} will produce \texttt{x + 0.5}, because by the time a SymPy
function sees the \texttt{1/2} it has already been converted to \texttt{0.5}
by Python. However, for a CAS like SymPy, we typically want to work with exact
rational numbers whenever possible. Working around this is simple, however:
one can wrap one of the integers with \texttt{Integer}, like \texttt{x +
  Integer(1)/2}, or using \texttt{x + Rational(1, 2)}. This gotcha is a small
downside to using Python directly instead of a custom DSL, and we consider it
to be worth it for the advantages listed above.

%%
%% Assumptions
\subsection{Assumptions}

A important feature of the SymPy core is the assumptions system. The
assumptions system allows users to specify that symbols have certain common
mathematical properties, such as being positive, imaginary, or integer. SymPy
is careful to never perform simplifications on an expression unless the
assumptions allow them. For instance, the identity $\sqrt{x^2} = x$ holds if
$x$ is nonnegative ($x>=0$). If $x$ is real, the identity $\sqrt{x^2}=|x|$
holds. However, for general complex $x$, no such identity holds.

By default, SymPy performs all calculations assuming that variables are
complex valued. This assumption makes it easier to treat mathematical problems
in full generality.

\begin{verbatim}
>>> x = Symbol('x')
>>> sqrt(x**2)
sqrt(x**2)
\end{verbatim}

By assuming symbols are complex by default, SymPy avoids performing
mathematically invalid operations. However, in many cases users will wish to
simplify expressions containing terms like $\sqrt{x^2}$.

Assumptions are set on \texttt{Symbol} objects when they are created. For
instance \texttt{Symbol('x', positive=True)} will create a symbol named
\texttt{x} that is assumed to be positive.

\begin{verbatim}
>>> x = Symbol('x', positive=True)
>>> sqrt(x**2)
x
\end{verbatim}

Some common assumptions that SymPy allows are \texttt{positive},
\texttt{negative}, \texttt{real}, \texttt{nonpositive}, \texttt{nonnegative},
\texttt{real}, \texttt{integer}, and \texttt{commutative}\footnote{If $A$ and
  $B$ are Symbols created with \texttt{commutative=False} then SymPy will keep
  $AB$ and $BA$ distinct.}. Assumptions on any object can be checked with the
\verb|is_|\texttt{\textit{assumption}} attribute.

Assumptions are needed only to restrict a domain so that certain
simplifications can be performed. It is not required to make the domain match
the input of a function. For instance, one can create the object
$\sum_{n=0}^m f(n)$ as \texttt{Sum(f(n), (n, 0, m))} without setting
\texttt{integer=True} when creating the Symbol object \texttt{n}.

The assumptions system additionally has deductive capabilities. The
assumptions use a three-valued logic using the Python builtin objects
\texttt{True}, \texttt{False}, and \texttt{None}. \texttt{None} represents the
``unknown'' case, meaning either the given assumption can be either true or
false under the given information, for instance, \verb|Symbol('x', real=True).is_real|
will give \texttt{None} because a real symbol might be
positive or it might not, or not enough is implemented to determine, for
instance, \verb|(pi + E).is_irrational| gives \texttt{None}, because SymPy
does not know how to determine if $\pi + e$ is rational or irrational.

Basic implications between the facts are used to deduce assumptions. For
instance, the assumptions system knows that being an integer implies being
rational, so \verb|Symbol('x', integer=True).is_rational| returns
\texttt{True}. Furthermore, expressions compute the assumptions on themselves
based on the assumptions of their arguments.

SymPy also has an experimental assumptions system where facts are stored
separate from objects, and deductions are made with a SAT solver. We will not
discuss this system here.

%%
%% Extensibility
\subsection{Extensibility}

%% Functions
\subsection{Functions}

Mathematical functions are implemented as Python classes.

%% Pattern matching
%%
%%%% three kinds of pattern matching in SymPy core.
%%%% `strategies` and `unify` modules also worth mentioning
%%%% (fu trig simplification function).
%%%% Rewriting rules.

%% Withheld evaluated (appears frequently on StackOverflow as a question).


\section{Numerics}
\label{sec:numerics}

%% Description of some algorithms (example: integration with Risch, Meijer G, Gruntz, polys)
%%
%% Description of numerics/mpmath (Fredrik)

\input{numerics}

\section{Features}
\label{sec:features}

%% List of Features and how to use
%%
%% Quick overview of the main modules, what it can do and so on. It should probably provide examples how to use sympy.
%%
%% See also the supplement (below)

% Features to discuss in-depth:

% Basic operations (the core)
\subsection{Basic Operations}
% Including the assumptions system

% Calculus (differentiation, integration, limits). Note that algorithm
% descriptions will go in the algorithms section.
\subsection{Calculus}

% Solvers (regular equations, maybe also mention other types of solvers like ODEs/recurrence/Diophantine)
\subsection{Solvers}

% Matrices (worth including to stress that they are symbolic)
\subsection{Matrices}

% Physics module (some sampling, to show that it is there)
\subsection{Physics}

% Series module (Formal Power Series, Fourier Series)
\subsection{Series}

% Series expansion (Differentiate between the two approaches being used)
\subsubsection{Series Expansion}

SymPy has several different implementations for representing series expansions.

\subsubsection{Formal Power Series}

SymPy can be used for computing the Formal Power Series of a function.
The implementation is based on the algorithm described in the paper on Formal Power Series\cite{Gruntz93formalpower}.
The advantage of this approach is that an explicit formula for the coefficients
of the series expansion is generated rather than just computing a few terms.

\begin{verbatim}
>>> from sympy import fps, symbols, sin
>>> x, y = symbols('x')
>>> f = fps(sin(x), x, x0=0)
>>> f.truncate(6)
x - x**3/6 + x**5/120 + O(x**6)
>>> f[15]
-x**15/1307674368000
\end{verbatim}

% Features to list, but not discuss in-depth:

% discrete math, concrete math, plotting, geometry, statistics, polys,
% combinatorics/group theory, code generation, tensors, lie algebras,
% cryptography, category theory, special functions, sets, matrix expressions,
% series, or vectors.
\subsection{Other features}


\section{Domain Specific Submodules}
\label{sec:domain_specific}

SymPy includes several packages that allow users to solve domain specific
problems. For example, a comprehensive physics package is included that is
useful for solving problems in classical mechanics, optics, and quantum
mechanics along with support for manipuating physical quantities with units.

\subsection{Vector Algebra}

The \verb|sympy.physics.vector| package provides reference frame, time, and
space aware vector and dyadic objects that allow for three dimensional
operations such as addition, subtraction, scalar multiplication, inner and
outer products, cross products, etc. Both of these objects can be written in
very compact notation that make it easy to express the vectors and dyadics in
terms of multiple reference frames with arbitrarily defined relative
orientations. The vectors are used to specify the positions, velocities, and
accelerations of points, orientations, angular velocities, and angular
accelerations of reference frames, and force and torques. The dyadics are
essentially reference frame aware $3 \times 3$ tensors. The vector and dyadic
objects can be used for any one-, two-, or three-dimensional vector algebra and
they provide a strong framework for building physics and engineering tools.

\begin{listing}
  \begin{minted}{pycon}
>>> from sympy import pi
>>> from sympy.physics.vector import ReferenceFrame
>>> A = ReferenceFrame('A')
>>> B = ReferenceFrame('B')
>>> C = ReferenceFrame('C')
>>> B.orient(A, 'body', (pi, pi / 3, pi / 4), 'zxz')
>>> C.orient(B, 'axis', (pi / 2, B.x))
>>> v = 1 * A.x + 2 * B.z + 3 * C.y
>>> v
A.x + 2*B.z + 3*C.y
>>> v.express(A)
A.x + 5*sqrt(3)/2*A.y + 5/2*A.z
  \end{minted}
  \caption{
    Python interpreter session showing how a vector is created using the
    orthogonal unit vectors of three reference frames that are oriented with
    respect to each other and the result of expressing the vector in the $A$
    frame.
    The $B$ frame is oriented with respect to the $A$ frame using Z-X-Z Euler
    Angles of magnitude $\pi$, $\frac{\pi}{2}$, and
    $\frac{\pi}{3}$\si{\radian},
    respectively whereas the $C$ frame is oriented with respect to the $B$
    frame through a simple rotation about the $B$ frame's X unit vector through
    $\frac{\pi}{2}$\si{\radian}.}
  \label{lis:physics-vector}
\end{listing}

\subsection{Classical Mechanics}

The \verb|physics.mechanics| package utilizes the \verb|physics.vector| package
to populate time aware particle and rigid body objects to fully describe the
kinematics and kinetics of a rigid multi-body system. These objects store all
of the information needed to derive the ordinary differential or differential
algebraic equations that govern the motion of the system, i.e., the equations
of motion. These equations of motion abide by Newton's laws of motion and can
handle any arbitrary kinematical constraints or complex loads. The package
offers two automated methods for formulating the equations of motion based on
Lagrangian Dynamics~\cite{Lagrange1811} and Kane's Method~\cite{Kane1985}. Lastly, there
are automated linearization routines for constrained dynamical
systems based on~\cite{Peterson2014}.

\subsection{Quantum Mechanics}

The \verb|sympy.physics.quantum| package provides quantum functions, states,
operators, and computation of standard quantum models.

% TODO : This needs some help from someone that knows something about quantum
% physics. I wasn't able to understand much from the documentation.

\subsection{Optics}

The \verb|physics.optics| package provides Gaussian optics functions.

% TODO : This needs some help from someone that knows something about optics.

\subsection{Units}

The \verb|physics.units| module provides around two hundred predefined prefixes
and SI units that are commonly used in the sciences. Additionally, it provides
the \verb|Unit| class which allows the user to define their own units.  These
prefixes and units are multiplied by standard SymPy objects to make expressions
unit aware, allowing for algebraic and calculus manipulations to be applied to
the expressions while the units are tracked in the manipulations.  The units of
the expressions can be easily converted to other desired units.  There is also
a new units system in \verb|sympy.physics.unitsystems| that allows the user to
work in specified unit systems.


\section{Conclusion and future work}
\label{sec:conclusion}

SymPy provides a wide array of features for symbolic algebra. It is designed
to be used in an extensible way, both as a end-user application and as a
library. It

The goal of SymPy is somewhat lofty: to be a full-featured CAS\@. However, a
few things sit at the head of the project's roadmap.

% Performance. Mention SymEngine.

%


\section{References}

\bibliographystyle{siamplain}
\bibliography{paper}

% Adding this here for now so it stays in one document. Will be a separate
% document at the end.
\section{Supplement}

\subsection{Limits: The Gruntz Algorithm}

SymPy calculates limits using the Gruntz algorithm, as described in%
~\cite{Gruntz1996limits}. The basic idea is as follows: any limit can be
converted to a limit $\lim\limits_{x\to\infty} f(x)$ by substitutions like
$x\to{1\over x}$. Then the most varying subexpression $\omega$ (that converges
to zero as $x\to\infty$ the fastest from all subexpressions) is identified in
$f(x)$, and $f(x)$ is expanded into a series with respect to $\omega$. Any
positive powers of $\omega$ converge to zero. If there are negative powers of
$\omega$, then the limit is infinite. The constant term (independent of
$\omega$, but could depend on $x$) then determines the limit (one might need to
recursively apply the Gruntz algorithm on this term to determine the limit).

To determine the most varying subexpression, the comparability classes must
first be defined, by calculating $L$:
\begin{equation}
L\equiv \lim_{x\to\infty} {\log |f(x)| \over \log |g(x)|}
\end{equation}
And then operations $<$, $>$ and $\sim$ are defined as follows: $f>g$ when
$L=\pm\infty$ (it is said that $f$ is more rapidly varying than $g$, i.e., $f$
goes to $\infty$ or $0$ faster than $g$, $f$ is greater than any power of $g$),
$f<g$ when $L=0$ ($f$ is less rapidly varying than $g$) and $f\sim g$ when
$L\neq 0,\pm\infty$ (both $f$ and $g$ are bounded from above and below by
suitable integral powers of the other). Here are some examples of comparability
classes:
$$2 < x < e^x < e^{x^2} < e^{e^x}$$
$$2\sim 3\sim -5$$
$$x\sim x^2\sim x^3\sim {1\over x}\sim x^m\sim -x$$
$$e^x\sim e^{-x}\sim e^{2x}\sim e^{x+e^{-x}}$$
$$f(x)\sim{1\over f(x)}$$

The Gruntz algorithm is now illustrated on the following example:
\begin{equation}
    \label{gruntz_example_fn}
f(x) = e^{x+2e^{-x}} - e^x + {1\over x} \,.
\end{equation}
The goal is to calculate $\lim\limits_{x\to\infty} f(x)$.
First the set of most rapidly varying subexpressions is determined, the so
called \textit{mrv set}. For~\eqref{gruntz_example_fn}, the following mrv set
$\{e^x, e^{-x}, e^{x+2e^{-x}}\}$ is obtained. These are all subexpressions of%
~\eqref{gruntz_example_fn} and they all belong to the same comparability class.
This calculation can be done using SymPy as follows:

\begin{verbatim}
>>> from sympy.series.gruntz import mrv
>>> mrv(exp(x+2*exp(-x))-exp(x) + 1/x, x)[0].keys()
dict_keys([exp(x + 2*exp(-x)), exp(x), exp(-x)])
\end{verbatim}

Next any item $\omega$ is taken from mrv that converges to zero for
$x\to\infty$. The item $\omega=e^{-x}$ is obtained. If such a term is not
present in the mrv set (i.e., all terms converge to infinity instead of zero),
the relation $f(x)\sim {1\over f(x)}$ can be used.

Next step is to rewrite the mrv in terms of $\omega$: $\{{1\over\omega},
\omega, {1\over\omega}e^{2\omega}\}$. Then the original subexpressions are
substituted back into $f(x)$ and expanded with respect to $\omega$:
\begin{equation}
    \label{gruntz_example_fn2}
f(x) = {1\over x}-{1\over\omega}+{1\over\omega}e^{2\omega}
     = 2+{1\over x} + 2\omega + O(\omega^2)
\end{equation}

Since $\omega$ is from the mrv set, then in the limit $x\to\infty$ it is
$\omega\to0$ and so $2\omega + O(\omega^2) \to 0$ in~\eqref{gruntz_example_fn2}:
\begin{equation}
f(x) = {1\over x}-{1\over\omega}+{1\over\omega}e^{2\omega}
    = 2+{1\over x} + 2\omega + O(\omega^2)
    \to 2 + {1\over x}
\end{equation}

Since the result ($2+{1\over x}$) still depends on $x$, the above procedure is
iterated on the result until just a number (independent of $x$) is obtained,
which is the final limit. In the above case the limit is $2$, as can be
verified by SymPy:

\begin{verbatim}
>>> limit(exp(x+2*exp(-x))-exp(x) + 1/x, x, oo)
2
\end{verbatim}

In general, when $f(x)$ is expanded in terms of $\omega$, it is obtained:
\begin{equation}
f(x) = \underbrace{O\left({1\over \omega^3}\right)}_\infty
    + \underbrace{C_{-2}(x)\over \omega^2}_\infty
    + \underbrace{C_{-1}(x)\over \omega}_\infty
    + {C_{0}(x)}
    + \underbrace{C_{1}(x)\omega}_0
    + \underbrace{O(\omega^2)}_0
\end{equation}
The positive powers of $\omega$ are zero. If there are any negative powers of
$\omega$, then the result of the limit is infinity, otherwise the limit is
equal to $\lim\limits_{x\to\infty} C_0(x)$. The expression $C_0(x)$ is simpler
than $f(x)$ and so the algorithm always converges. A proof of this, as well as
further details are given in Gruntz's Ph.D. thesis~\cite{Gruntz1996limits}.


% Series module (Formal Power Series, Fourier Series)
\subsection{Series}

% Series expansion (Differentiate between the two approaches being used)
\subsubsection{Series Expansion}

SymPy is able to calculate the symbolic series expansion of an arbitrary series
or expression involving elementary and special functions and multiple
variables. For this it has two different implementations- the \texttt{series}
method and Ring Series.

The first approach stores a series as an object of the \texttt{Basic} class.
Each function has its specific implementation of its expansion which is able to
evaluate the Puiseux series expansion about a specified point. For example,
consider a Taylor expansion about 0:

\begin{verbatim}
>>> from sympy import symbols, series
>>> x, y = symbols('x, y')
>>> series(sin(x+y) + cos(x*y), x, 0, 2)
1 + sin(y) + x*cos(y) + O(x**2)
\end{verbatim}

The newer and much faster\cite{sympyRingSeries} approach called Ring Series makes use of the
observation that a truncated Taylor series, is in fact a polynomial.
Ring Series uses the efficient representation and operations of sparse
polynomials. The choice of sparse polynomials is deliberate as it performs
well in a wider range of cases than a dense representation. Ring Series gives
the user the freedom to choose the type of coefficients he wants to have in
his series, allowing the use of faster operations on certain types.

For this, several low level methods for expansion of trigonometric, hyperbolic
and other elementary functions like inverse of a series, calculating $n$th
root, etc, are implemented using variants of the Newton\cite{zimmerman} Method.
All these support Puiseux series expansion. The following example demonstrates
the use of an elementary function that calculates the Taylor expansion of the
\texttt{sine} of a series.

\begin{verbatim}
>>> from sympy import ring
>>> from sympy.polys.ring_series import rs_sin
>>> R, t = ring('t', QQ)
>>> rs_sin(t**2 + t, t, 5)
-1/2*t**4 - 1/6*t**3 + t**2 + t
\end{verbatim}

The function \texttt{sympy.polys.rs\_series} makes use of these elementary
functions to expand an arbitrary SymPy expression. It does so by following a
recursive strategy of expanding the lower most functions first and then
composing them recursively to calculate the desired expansion. Currently it
only supports expansion about 0 and is under active development. Ring Series
is several times faster than the default implementation with the speed
difference increasing with the size of the series. The
\texttt{sympy.polys.rs\_series} takes as input any SymPy expression and hence
there is no need to explicitly create a polynomial \texttt{ring}. An example:

% no-doctest
\begin{verbatim}
>>> from sympy.polys.ring_series import rs_series
>>> from sympy.abc import a, b
>>> from sympy import sin, cos
>>> rs_series(sin(a + b), a, 4)
-1/2*(sin(b))*a**2 + (sin(b)) - 1/6*a**3*(cos(b)) + a*(cos(b))
\end{verbatim}

\subsubsection{Formal Power Series}

SymPy can be used for computing the Formal Power Series of a function.
The implementation is based on the algorithm described in the paper on
Formal Power Series\cite{Gruntz93formalpower}.  The advantage of this approach is
that an explicit formula for the coefficients of the series expansion is generated
rather than just computing a few terms.

The following example shows how to use \texttt{fps}:

\begin{verbatim}
>>> f = fps(sin(x), x, x0=0)
>>> f.truncate(6)
x - x**3/6 + x**5/120 + O(x**6)
>>> f[15]
-x**15/1307674368000
\end{verbatim}

\subsubsection{Fourier Series}

SymPy provides functionality to compute Fourier Series of a function using
the \texttt{fourier\_series} function. Under the hood it just computes $a0$, $an$, $bn$ using
standard integration formulas.

Here's an example on how to compute Fourier Series in SymPy:

\begin{verbatim}
>>> L = symbols('L')
>>> f = fourier_series(2 * (Heaviside(x/L) - Heaviside(x/L - 1)) - 1, (x, 0, 2*L))
>>> f.truncate(3)
4*sin(pi*x/L)/pi + 4*sin(3*pi*x/L)/(3*pi) + 4*sin(5*pi*x/L)/(5*pi)
\end{verbatim}


% Logic module
\subsection{Logic}

\input{logic}

\subsection{Diophantine Equations}

Diophantine equations play a central and an important role in number theory.
A Diophantine equation has the form, $f(x_1, x_2, \ldots x_n) = 0$
where $n \geq 2$ and $x_1, x_2, \ldots x_n$ are integer variables. If we can find
$n$ integers $a_1, a_2, \ldots a_n$ such that $x_1 = a_1, x_2 = a_2, \ldots x_n = a_n$
satisfies the above equation, we say that the equation is solvable.

Currently, following five types of Diophantine equations can be solved using
SymPy's Diophantine module.

\begin{itemize}
    \item Linear Diophantine equations: $a_1x_1 + a_2x_2 + \cdots + a_{n}x_{n} = b$
    \item General binary quadratic equation: $ax^2 + bxy + cy^2 + dx + ey + f = 0$
    \item Homogeneous ternary quadratic equation: $ax^2 + by^2 + cz^2 + dxy + eyz + fzx = 0$
    \item Extended Pythagorean equation: $a_{1}x_{1}^2 + a_{2}x_{2}^2 + \cdots + a_{n}x_{n}^2 = a_{n+1}x_{n+1}^2$
    \item General sum of squares: $x_{1}^2 + x_{2}^2 + \cdots + x_{n}^2 = k$
\end{itemize}

When an equation is fed into Diophantine module, it factors the equation (if
possible) and solves each factor separately. Then all the results are combined to
create the final solution set. Following examples illustrate some of the basic
functionalities of the Diophantine module.

\begin{verbatim}
>>> from sympy import symbols
>>> x, y, z = symbols("x, y, z", integer=True)

>>> diophantine(2*x + 3*y - 5)
set([(3*t_0 - 5, -2*t_0 + 5)])

>>> diophantine(2*x + 4*y - 3)
set()

>>> diophantine(x**2 - 4*x*y + 8*y**2 - 3*x + 7*y - 5)
set([(2, 1), (5, 1)])

>>> diophantine(x**2 - 4*x*y + 4*y**2 - 3*x + 7*y - 5)
set([(-2*t**2 - 7*t + 10, -t**2 - 3*t + 5)])

>>> diophantine(3*x**2 + 4*y**2 - 5*z**2 + 4*x*y - 7*y*z + 7*z*x)
set([(-16*p**2 + 28*p*q + 20*q**2, 3*p**2 + 38*p*q - 25*q**2, 4*p**2 - 24*p*q + 68*q**2)])

>>> from sympy.abc import a, b, c, d, e, f
>>> diophantine(9*a**2 + 16*b**2 + c**2 + 49*d**2 + 4*e**2 - 25*f**2)
set([(70*t1**2 + 70*t2**2 + 70*t3**2 + 70*t4**2 - 70*t5**2, 105*t1*t5, 420*t2*t5, 60*t3*t5, 210*t4*t5, 42*t1**2 + 42*t2**2 + 42*t3**2 + 42*t4**2 + 42*t5**2)])

>>> diophantine(a**2 + b**2 + c**2 + d**2 + e**2 + f**2 - 112)
set([(8, 4, 4, 4, 0, 0)])
\end{verbatim}


% Sets
\subsection{Sets}
%% Sets SymPy

SymPy supports representation of a wide variety of set, this is achieved by
first defining abstract representation for a smaller number of sets in form of
python classes and then combining and transforming them using various set
operations.

%% Set Types

\begin{itemize}
    \item Finite Set
    \item Empty Set
    \item Universal Set
    \item Real Interval
    \item Finite and Infinite Range
    \item Complex Region
\end{itemize}


%% Operations

\begin{itemize}
    \item Finite Union
    \item Finite Intersection
    \item Set Difference
    \item Product Set
    \item Image Set
    \item Condition Set
\end{itemize}

%% Representations acheivable through application of Operations on atomic set
%% types mentioned above.


%% Special Cases


\subsection{Category Theory}
SymPy includes a basic version of the module for dealing with
categories — abstract mathematical objects representing classes of
structures as classes of objects (points) and morphisms (arrows)
between the objects.  This version of the module was designed with the
following two goals in mind:

\begin{enumerate}
\item automatic typesetting of diagrams given by a collection of
  objects and of morphisms between them, and
\item specification and (semi-)automatic derivation of properties
  using commutative diagrams.
\end{enumerate}

At the time of writing of this paper, the version in the {\tt master}
branch only implements the first goal, while a (very partially
working) draft of implementation of the second goal is available at
\url{https://github.com/scolobb/sympy/tree/ct4-commutativity}.

In order to achieve the two goals, the module {\tt categories} defines
several classes representing some of the essential concepts: objects,
morphisms, categories, diagrams.  Since in category theory the inner
structure of its objects is often discarded (in the favour of studying
the properties of morphisms), the class {\tt Object} is essentially a
synonym of the class {\tt Symbol}.  There are several morphism classes
which do not have a particular internal structure either, except for
{\tt CompositeMorphism}, which essentially stores a list of morphisms.
To capture the properties of morphisms, the class {\tt Diagram} is
expected to be used.  This class stores a family of morphisms, the
corresponding source and target objects, and, possibly, some
properties of the morphisms.  Generally, no restrictions are imposed
on what the properties may be --- for example, one might use strings
of the form ``forall'', ``exists'', ``unique'', etc.  Furthermore, the
morphisms of a diagram are grouped into {\em premises} and {\em
  conclusions}, in order to be able to represent logical implications
of the form ``for a collection of mophisms $P$ with properties $p:P\to
\Omega$ (the premises), there exists a collection of morphisms $C$
with properties $c:C\to \Omega$ (the conclusions)'', where $\Omega$ is
the universal collection of properties.  Finally, the class {\tt
  Category} includes a collection of diagrams which are deemed
commutative and which therefore define the properties of this
category.

Automatic typesetting of diagrams takes a {\tt Diagram} and produces
\LaTeX~code using the {\tt Xy-pic} package.  Typesetting is done in
two stages: layout and generation of {\tt Xy-pic} code.  The layout
stage is taken care of by the class {\tt DiagramGrid} which takes a
{\tt Diagram} and lays out the objects in a grid, trying to reduce the
average length of the arrows in the final picture.  By default, {\tt
  DiagramGrid} uses a series of triangle-based heuristics to produce a
rectangular grid.  A linear layout can also be imposed.  Furthermore,
groups of objects can be given; in this case, the groups will be
treated as atomic cells, and the member objects will be typeset
independently of the other objects.

The second phase of diagram typesetting consists in actually drawing
the picture and is carried out by the class {\tt XypicDiagramDrawer}.
An example of a diagram automatically typeset by {\tt DiagramgGrid}
and {\tt XypicDiagramDrawer} in given in Figure~\ref{fig:cat:loops}.
\begin{figure}[h]
  \centerline{
    \xymatrix{
      A \ar[r]_{f} \ar@/^3mm/[rr]^{h_{2}} \ar@(u,l)[]^{l_{A}} \ar@/^3mm/@(l,d)[]^{n_{A}} & B \ar[d]^{g} & D \ar[l]^{k} \ar@/_7mm/[ll]_{h} \ar@/_11mm/[ll]_{h_{1}} \ar@(r,u)[]^{l_{D}} \ar@/^3mm/@(d,r)[]^{n_{D}} \\
      & C \ar@(l,d)[]^{l_{C}} \ar@/^3mm/@(d,r)[]^{n_{C}} &
    }
  }
  \caption{An automatically typeset commutative diagram}
  \label{fig:cat:loops}
\end{figure}

As far as the second main goal of the module is concerned, a
(non-working) draft of an implementation is in
\url{https://github.com/scolobb/sympy/tree/ct4-commutativity}.  The
principal idea consists in automatically deciding whether a diagram is
commutative or not, given a collection of ``axioms'' --- diagrams {\em
  known} to be commutative.  The approach to implementation is based
on graph embeddings (injective maps): whenever an embedding of a
commutative diagram into a given diagram is found, one concludes that
that subdiagram is commutative.  Deciding commutativity of the whole
diagram is therefore based (theoretically) on finding a ``cover'' of
the target diagram by embeddings of the axioms.  The naïve
implementation proved to be prohibitively slow; a better optimised
version is therefore in order, as well as application of heuristics.

Contributions to automatic inference of commutativity of diagrams are
welcome.  The source code (both the one in master and in {\tt
  ct4-commutativity}) is extensively documented.  Even more extensive
explanations (including some literary chatter) are given in
\url{https://scolobb.wordpress.com/}.


\subsection{SymPy Gamma}\label{sympy-gamma}


SymPy Gamma is a simple web application that runs on Google App Engine.
It executes and displays the results of SymPy expressions as well as
additional related computations, in a fashion similar to that of
Wolfram\textbar{}Alpha. For instance, entering an integer will display
its prime factors, digits in the base-10 expansion, and a factorization
diagram. Entering a function will display its docstring; in general,
entering an arbitrary expression will display its derivative, integral,
series expansion, plot, and roots.

SymPy Gamma also has several additional features than just computing the
results using SymPy.

\begin{itemize}
\item
  It displays integration steps, differentiation steps in detail, which
  can be viewed in Figure~\ref{fig:integralsteps}:\par
\begin{minipage}{\textwidth}
    \centering
    \includegraphics[width=0.7\textwidth]{integral_steps.png}
    \captionof{figure}{Integral steps of $\tan (x)$}\label{fig:integralsteps}
\end{minipage}
\item
  It also displays the factor tree diagrams for different numbers.
\item
  SymPy Gamma also saves user search queries, and offers many such
  similar features for free, which Wolfram\textbar{}Alpha only offers
  to its paid users.
\end{itemize}
Every input query from the user on SymPy Gamma is first, parsed by its
own parser, which handles several different forms of function names,
which SymPy as a library doesn't support. For instance, SymPy Gamma
supports queries like \texttt{sin\ x}, whereas SymPy doesn't support
this, and supports only \verb|sin(x)|.

This parser converts the input query to the equivalent SymPy readable
code, which is then eventually processed by SymPy and the result is
finally formatted in LaTeX and displayed on the SymPy Gamma web-application.


\subsection{SymPy Live}\label{sympy-live}


SymPy Live is an online Python shell, which runs on Google
App Engine, that executes SymPy code. It is integrated in the SymPy
documentation examples, located at this \href{http://docs.sympy.org/latest/index.html}{link}.

This is accomplished by providing a HTML/JavaScript GUI for entering
source code and visualization of output, and a server part which
evaluates the requested source code. It's an interactive AJAX shell,
that runs SymPy code using Python on the server.
\newline
Certain Features of SymPy Live:

\begin{itemize}
\item
  It supports the exact same syntax as SymPy, hence it can be used
  easily, to test for outputs of various SymPy expressions.
\item
  It can be run as a standalone app or in an existing app as an
  admin-only handler, and can also be used for system administration
  tasks, as an interactive way to try out APIs, or as a debugging aid
  during development.
\item
  It can also be used to plot figures (\href{http://live.sympy.org/?evaluate=from\%20sympy\%20import\%20symbols\%0Afrom\%20sympy.plotting\%20import\%20textplot\%0Ax\%20\%3D\%20symbols(\%27x\%27)\%0Atextplot(x**2\%2C0\%2C5)\%0A\%23--\%0A}{link}),
  and execute all kinds of expressions that SymPy can evaluate.
\item
SymPy Live also formats the output in LaTeX for pretty-printing the
output.
\end{itemize}


\subsection{Comparison with Mathematica}


\subsection{Mathematica}

Wolfram Mathematica is a popular proprietary CAS.
It features highly advanced algorithms.
Mathematica has a core implemented in C++~\cite{Wolfram2016}
which interprets its own programming language (know as Wolfram language).

% M-expressions

Analogously to Lisp's S-expressions,
Mathematica uses its own style of M-expressions,
which are arrays of either atoms or other M-expression.
The first element of the expression identifies the type of the expression
and is indexed by zero, whereas the first argument is indexed by one.
Notice that SymPy expression arguments are stored in a Python tuple
(that is, an immutable array),
while the expression type is identified by the type of the object storing the
expression.

% Attributes

Mathematica can associate attributes to its atoms.

% Expression mutability

Unlike SymPy, Mathematica's expressions are mutable,
that is one can change parts of the expression tree without the need of
creating a new object.
The reactivity of Mathematica allows for a lazy updating of any references
to that data structure.

% * comparison with Mathematica: commutativity, associative expressions, one-identity. Advantage of SymPy: multiplicative commutativity defined on symbols.
% Products and commutativity

Products in Mathematica are determined by some builtin node types,
such as \texttt{Times}, \texttt{Dot}, and others.
\texttt{Times} is overloaded by the * operator,
and is always meant to represent a commutative operator.
The other notable product is \texttt{Dot}, overloaded by the . operator.
This product represents matrix multiplication,
it is not commutative.
SymPy uses the same node for both scalar and matrix multiplication,
the only exception being with abstract matrix symbols.
Unlike Mathematica, SymPy determines commutativity with respect to
multiplication from the factor's expression type.
Mathematica puts the \texttt{Orderless} attribute on the expression
type.

% Associative expressions.

Regarding associative expressions,
SymPy handles associativity by making associative expressions inherit the
class \texttt{AssocOp},
while Mathematica specifies the \texttt{Flat}\cite{WolframRefFlat} attribute on the expression type.

% One identity


% Pattern matching

Mathematica relies heavily on pattern matching:
even the so-called equivalent of function declaration is in reality
the definition of a pattern matching generating an expression tree transformation
on input expressions.
%
Mathematica's pattern matching is sensitive to associative\cite{WolframRefFlat}, commutative\cite{WolframRefOrderless},
and one-identity\cite{WolframRefOneIdentity} properties of its expression tree nodes\cite{WolframRefFlatAndOrderlessFunctions}.
%
SymPy has various ways to perform pattern matching. 
All of them play a lesser role in the CAS than in Mathematica.
looks less mature than Mathematica's,

%% TODO list:
% * comparison with Mathematica: MatrixExp, product not always commutative, type inheritance (polymorphism) and advantage in unifying the product symbol * for symbols and matrices, pattern matching vs. single dispatch.

% Type inheritance and polymorphism

Unlike SymPy, Mathematica does not support type inheritance or polymorphism\cite{Fateman1992}.
% cite examples of class inheritance in SymPy:
% 
SymPy relies heavily on class inheritance, but for the most part,
class inheritance is used to make sure that SymPy objects inherit the proper
methods and implement the basic hashing system.
Associativity of expressions can be achieved by inheriting the class \texttt{AssocOp},
which may appear a more cumbersome operation than Mathematica's attribute setting.
%There are also cases where inheritance is used to extend the mathematical meaning of an expression.

% Matrices

Matrices in SymPy are types on their own.
In Mathematica, nested lists are interpreted as matrices whenever the sublists
have the same length.
The main difference to SymPy is that ordinary operators and functions 
do not get generalized the same way as used in traditional mathematics.
Using the standard multiplication in Mathematica performs an elementwise
product, this is compatible with Mathematica's convention of commutativity of
\texttt{Times} nodes. 
Matrix product is expressed by the \textit{dot} operator, 
or the \texttt{Dot} node.
The same is true for the other operators, and even functions,
most notably calling the exponential function \texttt{Exp} on a matrix 
returns an elementwise exponentiation of its elements.
The real matrix exponentiationl is available through the \texttt{MatrixExp}
function.

% * comparison with Mathematica: avoid misspelling variables through forced declaration (check that you can't do it in Mathematica).
% * evaluate=False vs HoldForm

Unevaluated expressions can be achieved in various ways,
most commonly with the \texttt{HoldForm} or \texttt{Hold} nodes,
that block the evaluation of subnodes by the parser.
Note that such a node cannot be expressed in Python, because of greedy evaluation.
Whenever needed in SymPy, it is necessary to add the parameter \texttt{evaluate=False}
to all subnodes, or put the input expression in a string.

% * comparison with Mathematica: == is structural equality, not 

The operator == returns a boolean whenever it is able to immediately evaluate
the truthness of the equality, otherwise it returns an \texttt{Equal} expression.
In SymPy == means structural equality and is always guaranteed to return a
boolean expression.
To express an equality in SymPy it is necessary to explicitly construct the 
\texttt{Equality} class.

% * comparison with Mathematica: polynomial module.
% * comparison with Mathematica: space is product, ** vs ^

SymPy, in accordance with Python and unlike the usual programming convention,
uses ** to express the power operator, while Mathematica uses the more 
common ^.

% * comparison with Mathematica: ( ) is Sequence, functions are generally uppercase.
% * comparsion with Mathematica: table of comparison?
% * comparison with Mathematica: Wolfram language has loads of operator overloading, functional paradigm.



\subsection{Other Projects that use SymPy}

\subsection{SymPy Gamma}\label{sympy-gamma}

SymPy Gamma is a simple web application based on Google App Engine that 
executes and displays the results of SymPy expressions as well as
additional related computations, in a fashion similar to that of
Wolfram\textbar{}Alpha. For instance, entering an integer will display
prime factors, digits in the base-10 expansion, and a factorization
diagram. Entering a function will give its docstring; in general,
entering an arbitrary expression will provide the derivative, integral,
series expansion, plot, and roots.

SymPy Gamma also has several additional features than just computing the
results using SymPy.

\begin{itemize}
\item
  It displays integration steps, differentiation steps in detail, which
  can be viewed through this
  \href{http://www.sympygamma.com/input/?i=integrate\%281\%20/\%28\%28x\%2B1\%29\%28x\%2B3\%29\%28x\%2B5\%29\%29\%29}{Link}.
\item
  It also displays the factor tree diagrams for numbers, which is also
  illustrated through this
  \href{http://www.sympygamma.com/input/?i=112}{link}.
\item
  SymPy Gamma also saves user search queries, and offers many such 
  similar features for free, which Wolfram\textbar{}Alpha only offers 
  to it's paid users.
\end{itemize}
Every input query from the user on SymPy Gamma is first, parsed by it's
own parser, which handles several different forms of function names,
which SymPy as a library doesn't support. For instance, SymPy Gamma
supports queries like - \texttt{sin\ x}, whereas SymPy doesn't support
this, and supports only \texttt{sin(x)}.

This parser, then converts the input query to the resultant SymPy
readable code, which is then eventually processed by SymPy and the
result is finally formatted in LaTeX and displayed on the SymPy Gamma
web-app.


\subsection{Tensors}

Ongoing work to provide the capabilities of tensor computer algebra has so far
produced the \verb|tensor| module.  It is composed of three separated
submodules, whose purposes are quite different: \verb|tensor.indexed| and
\verb|tensor.indexed_methods| support indexed symbols,
\verb|tensor.array| contains facilities to operator on symbolic $N$-dimensional
arrays and finally \verb|tensor.tensor| is used to defineabstract tensors.
The abstract tensors subsection
is inspired by xAct\cite{xAct} and Cadabra\cite{Peeters2007cadabra}.
Canonicalization based on the Butler-Portugal\cite{ManssurPortugal1999}
algorithm is supported in SymPy.  It is currently limited to polynomial tensor
expressions.



\end{document}
