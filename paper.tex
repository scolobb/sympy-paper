\documentclass{article}
\usepackage{hyperref}
\usepackage{graphicx}
\usepackage{amsmath}
\usepackage{caption}
\graphicspath{ {images/} }

\usepackage{amsmath}
\usepackage{url}
\usepackage{hyperref}

% for nice source code syntax highlighting, also provides Listing env
\usepackage{minted}

% this is required for all the \url{} commands in the bib file
%\usepackage{hyperref}

% for nice units
\usepackage{siunitx}

% for images: png, pdf, etc
\usepackage{graphicx}

% for nice table formatting, i.e., /toprule, /midrule, etc
\usepackage{booktabs}

% to allow for \verb++ declarations in captions.
\usepackage{cprotect}

% to allow usage of \mathbb symbols
\usepackage{amssymb}

\usepackage{longtable}

\title{SymPy: Symbolic Computing in Python}

\begin{document}
\maketitle

\section{Introduction}

%% What sympy is, where to download etc.
%%
%% List other major CASs.
%%
%% Why SymPy.

SymPy is a full featured computer algebra system (CAS) written in the Python
programming language. It is open source, licensed under the extremely
permissive 3-clause BSD license, which allows anyone to reuse SymPy or its source code,
even for commercial purposes. SymPy was started by Ond\v{r}ej \v{C}ert\'{\i}k
in 2005, and it has since grown into a large open source project, with over
500 contributors.
% citation?

SymPy is written entirely in the Python programming language,
% cite Python?
which is also the language used to interact with it, both programmatically and
interactively. Python is a popular dynamically typed programming language that
has a focus on ease of use and readability. It also a very popular language
for scientific computing and data science, with a wide range of useful
libraries.
% Cite numpy, scipy, pandas
% We could also cite
% https://stackoverflow.com/research/developer-survey-2016#most-popular-technologies-per-occupation
Unlike many CASs, SymPy does not invent its own programming language. Python
is used both for the internal implementation and the user interaction.
Exclusively using Python in this way makes it easier for people already
familiar with the language to use SymPy, or to do development on it. It also
lets the SymPy developers focus on mathematics, rather than language design.

SymPy is designed with a strong focus that it be usable as a library. This
means that extensibility is important in its API design. This is also one of
the reasons SymPy makes no attempt to extend the Python language itself. The
goal is for users of SymPy to be able to import SymPy alongside other Python
libraries in their workflow, whether that is an interactive workflow or
programmatic use as part of a larger library.
% cite https://www.python.org/dev/peps/pep-0020/

SymPy does not have a built in graphical user interface (GUI), however, when
used in the Jupyter Notebook
% citation
SymPy expressions will pretty print using MathJax.
% citation


\section{Architecture}


%% I volunteer to write this section. --Aaron
%%
%% Representing symbolic expressions using Python objects

\subsection{The Core}

Every symbolic expression in SymPy is an instance of a Python class.
Expressions are represented by expression trees. The operators is represented
by the type of an expression and the child nodes are stored in the
\texttt{args} attribute. A leaf node in the expression tree has an empty
\texttt{args}.
The \texttt{args} attribute is provided by the class \texttt{Basic},
which is a superclass of all SymPy objects and
provides common methods to all SymPy tree-elements.
For example, take the expression $xy + 2$.

\begin{verbatim}
>>> from sympy import *
>>> x, y = symbols('x y')
>>> expr = x*y + 2
\end{verbatim}

The expression \texttt{expr} is an addition, so it is of type \texttt{Add}. The child
nodes of \texttt{expr} are \texttt{x*y} and \texttt{2}.

\begin{verbatim}
>>> type(expr)
<class 'sympy.core.add.Add'>
>>> expr.args
(2, x*y)
\end{verbatim}

We can dig further into the expression tree to see the full expression. For
example, the first child node, given by \texttt{expr.args[0]} is
\texttt{2}. Its class is \texttt{Integer}, and it has empty \texttt{args},
indicating that it is a leaf node

\begin{verbatim}
>>> expr.args[0]
2
>>> type(expr.args[0])
<class 'sympy.core.numbers.Integer'>
>>> expr.args[0].args
()
\end{verbatim}

The function \texttt{srepr} give a string representing a valid Python code,
containing all the nested class constructor calls to create the given expression:

\begin{verbatim}
>>> srepr(expr)
"Add(Mul(Symbol('x'), Symbol('y')), Integer(2))"
\end{verbatim}

Every SymPy expression satisfies a key invariant, namely,
\texttt{expr.func(*expr.args) == expr}. This means that expressions are
rebuildable from their \texttt{args}\footnote{\texttt{expr.func} is used
  instead of \texttt{type(expr)} to allow the function of an expression to be
  distinct from its actual Python class. In most cases the two are the same.}.
Here, we note that in SymPy, the \texttt{==} operator represents exact
structural equality, not mathematical equality. This allows us to test if any
two expressions are equal to one another as expression trees.

Python allows classes to overload operators. The Python interpreter translates
above \texttt{x*y + 2} to, roughly,
\verb|(x.__mul__(y)).__add__(2)|. \texttt{x} and \texttt{y}, returned from
the \texttt{symbols} function, are \texttt{Symbol} instances. The \texttt{2}
in the expression is processed by Python as a literal, and is stored as
Python's builtin \texttt{int} type. When \texttt{2} is called by the
\verb|__add__|, it is converted to the SymPy type \texttt{Integer(2)}. In
this way, SymPy expressions can be built in the natural way using Python
operators and numeric literals.

One must be careful in one particular instance. Python does not have a builtin
rational literal type. Given a fraction of integers in \texttt{1/2}, Python
will perform floating point division and produce \texttt{0.5}\footnote{This is
  the behavior in Python 3. In Python 2, \texttt{1/2} will perform integer
  division and produce \texttt{0}, unless one uses \texttt{from \_\_future\_\_
    import division}}. Because Python uses eager evaluation, expressions like
\texttt{x + 1/2} will produce \texttt{x + 0.5}, because by the time a SymPy
function sees the \texttt{1/2} it has already been converted to \texttt{0.5}
by Python. However, for a CAS like SymPy, we typically want to work with exact
rational numbers whenever possible. Working around this is simple, however:
one can wrap one of the integers with \texttt{Integer}, like \texttt{x +
  Integer(1)/2}, or using \texttt{x + Rational(1, 2)}. This gotcha is a small
downside to using Python directly instead of a custom DSL, and we consider it
to be worth it for the advantages listed above.

%%
%% Assumptions
\subsection{Assumptions}

A important feature of the SymPy core is the assumptions system. The
assumptions system allows users to specify that symbols have certain common
mathematical properties, such as being positive, imaginary, or integer. SymPy
is careful to never perform simplifications on an expression unless the
assumptions allow them. For instance, the identity $\sqrt{x^2} = x$ holds if
$x$ is nonnegative ($x>=0$). If $x$ is real, the identity $\sqrt{x^2}=|x|$
holds. However, for general complex $x$, no such identity holds.

By default, SymPy performs all calculations assuming that variables are
complex valued. This assumption makes it easier to treat mathematical problems
in full generality.

\begin{verbatim}
>>> x = Symbol('x')
>>> sqrt(x**2)
sqrt(x**2)
\end{verbatim}

By assuming symbols are complex by default, SymPy avoids performing
mathematically invalid operations. However, in many cases users will wish to
simplify expressions containing terms like $\sqrt{x^2}$.

Assumptions are set on \texttt{Symbol} objects when they are created. For
instance \texttt{Symbol('x', positive=True)} will create a symbol named
\texttt{x} that is assumed to be positive.

\begin{verbatim}
>>> x = Symbol('x', positive=True)
>>> sqrt(x**2)
x
\end{verbatim}

Some common assumptions that SymPy allows are \texttt{positive},
\texttt{negative}, \texttt{real}, \texttt{nonpositive}, \texttt{nonnegative},
\texttt{real}, \texttt{integer}, and \texttt{commutative}\footnote{If $A$ and
  $B$ are Symbols created with \texttt{commutative=False} then SymPy will keep
  $AB$ and $BA$ distinct.}. Assumptions on any object can be checked with the
\verb|is_|\texttt{\textit{assumption}} attribute.

Assumptions are needed only to restrict a domain so that certain
simplifications can be performed. It is not required to make the domain match
the input of a function. For instance, one can create the object
$\sum_{n=0}^m f(n)$ as \texttt{Sum(f(n), (n, 0, m))} without setting
\texttt{integer=True} when creating the Symbol object \texttt{n}.

The assumptions system additionally has deductive capabilities. The
assumptions use a three-valued logic using the Python builtin objects
\texttt{True}, \texttt{False}, and \texttt{None}. \texttt{None} represents the
``unknown'' case, meaning either the given assumption can be either true or
false under the given information, for instance, \verb|Symbol('x', real=True).is_real|
will give \texttt{None} because a real symbol might be
positive or it might not, or not enough is implemented to determine, for
instance, \verb|(pi + E).is_irrational| gives \texttt{None}, because SymPy
does not know how to determine if $\pi + e$ is rational or irrational.

Basic implications between the facts are used to deduce assumptions. For
instance, the assumptions system knows that being an integer implies being
rational, so \verb|Symbol('x', integer=True).is_rational| returns
\texttt{True}. Furthermore, expressions compute the assumptions on themselves
based on the assumptions of their arguments.

SymPy also has an experimental assumptions system where facts are stored
separate from objects, and deductions are made with a SAT solver. We will not
discuss this system here.

%%
%% Extensibility
\subsection{Extensibility}

%% Functions
\subsection{Functions}

Mathematical functions are implemented as Python classes.

%% Pattern matching
%%
%%%% three kinds of pattern matching in SymPy core.
%%%% `strategies` and `unify` modules also worth mentioning
%%%% (fu trig simplification function).
%%%% Rewriting rules.

%% Withheld evaluated (appears frequently on StackOverflow as a question).


\section{Algorithms}

%% Description of some algorithms (example: integration with Risch, Meijer G, Gruntz, polys)
%%
%% Description of numerics/mpmath (Fredrik)

% A description of some of the algorithms in SymPy. The list is not
% exhaustive.

% The sections here are preliminary. We may end up needing to cut some of
% this.

% XXX: Perhaps this should just be integrated into the features section.

\subsection{Numerics}

The \texttt{Float} class holds an arbitrary-precision binary floating-point value
and a precision in bits. An operation between two \texttt{Float}
inputs is rounded to the larger of the two precisions.

The preferred way to evaluate an expression numerically is with the
\texttt{evalf} method, which internally estimates the number of accurate
bits of the floating-point
approximation for each sub-expression, and adaptively increases the
working precision until the estimated accuracy of the
final result matches the sought number of decimal digits.

The internal error tracking does not provide rigorous error bounds
(in the sense of interval arithmetic) and cannot be used to track
uncertainty in measurement data in any meaningful way;
the sole purpose is to mitigate loss of accuracy that typically occurs
when converting symbolic expressions to numerical values, for example
due to catastrophic cancellation. This is illustrated by the following
example:

\begin{verbatim}
>>> cos(exp(-100)).evalf(25) - 1
0
>>> (cos(exp(-100)) - 1).evalf(25)
-6.919482633683687653243407e-88
\end{verbatim}

The numerical evaluation works with complex numbers and supports
more complicated expressions, such as
higher special functions, infinite series and integrals.

[todo: contrast with mathematica's significance arithmetic? cite Sofroniou2005precise, maybe Fateman]

\subsubsection{The mpmath library}

The implementation of arbitrary-precision floating-point arithmetic
is supplied by the mpmath library, which originally was developed
as a SymPy module but subsequently has been
moved to a standalone Python package.

Like SymPy, mpmath is a pure Python library.
Internally, mpmath represents a floating-point number
$(-1)^s x \cdot 2^y$ by a tuple $(s, x, y, b)$ where
$x$ and $y$ are arbitrary-size Python integers
and the redundant integer $b$ stores the bit length of $x$ for quick access.
If GMPY [citation?] is installed, mpmath automatically switches to
using the \texttt{gmpy.mpz} type for $x$ and using GMPY helper methods
to perform rounding-related operations, improving performance.

The mpmath library includes support for complex numbers, evaluation of
special functions, root-finding, linear algebra, polynomial approximation,
and numerical computation of limits, derivatives, integrals, infinite
series, and ODE solutions. All features work in arbitrary precision
and use algorithms that support computig hundreds of digits rapidly,
except in degenerate cases.

The double exponential (tanh-sinh) quadrature is used for numerical
integration by default. For smooth integrands, this algorithm usually
converges extremely rapidly, even when the integration interval is infinite
or singularities are present at the endpoints.
However, for good performance, singularities
in the middle of the interval must be specified
the user.
To evaluate slowly converging limits and infinite series, mpmath
automatically attempts to apply Richardson extrapolation and the
Shanks transformation (Euler-Maclaurin summation can also be used).
A function to evaluate oscillatory integrals by means of convergence
acceleration is also available.

A wide array of higher mathematical functions are implemented
with full support for complex values of all parameters and arguments,
including complete and incomplete gamma functions,
Bessel functions, orthogonal polynomials, elliptic functions and integrals,
zeta and polylogarithm functions,
the generalized hypergeometric function, and the Meijer G-function.

Most special functions are implemented as linear 
combinations of the generalized hypergeometric function ${}_pF_q$,
which is computed by a combination of direct summation,
argument transformations (for ${}_2F_1$, ${}_3F_2$, $\ldots$)
and asymptotic expansions
(for ${}_0F_1$, ${}_1F_1$, ${}_1F_2$, ${}_2F_2$, ${}_2F_3$)
to cover the whole complex domain.
Numerical integration and generic convergence acceleration
are also used in a few special cases.

In general, linear combinations and argument transformations
give rise to singularities that have to be removed for certain
combinations of parameters.
A good example is the modified Bessel function of the second kind
$$K_{\nu}(z) = \frac{1}{2} \left[
            \left(\frac{z}{2}\right)^{-\nu}
                \Gamma(\nu)
                {}_0F_1\left(1-\nu, \frac{z^2}{4}\right)
             -
             \left(\frac{z}{2}\right)^{\nu}
                 \frac{\pi}{\nu \sin(\pi \nu) \Gamma(\nu)}
                 {}_0F_1\left(\nu+1, \frac{z^2}{4}\right)
            \right]$$
where the limiting value $\lim_{\varepsilon \to 0} K_{n+\varepsilon}(z)$
has to be computed when $\nu = n$ is an integer.
A generic algorithm is used to evaluate
hypergeometric-type linear combinations of the above type.
The implementation automatically detects cancellation problems,
and computes limits numerically by perturbing parameters whenever
internal singularities occur (the perturbation size is automatically
decreased until the result is detected to converge numerically).

It is important for a computer algebra system
to support generalized hypergeometric functions and
Meijer G-functions robustly, due to their frequent appearance
in closed forms for integrals and sums.
Via mpmath, SymPy has relatively good support for evaluating sums and integrals
numerically, using two complementary approaches: direct numerical evaluation,
or first computing a symbolic closed form involving special functions. [example?]

\subsubsection{Numerical simplification}

The \texttt{nsimplify} method attempts to find a simple symbolic
expression that evaluates to the same numerical value as the given
input. It is a wrapper of the \texttt{identify} method in mpmath.
It works by applying a few simple transformations
(including square roots, reciprocals, logarithms and exponentials) to
the input and, for each transformed value,
using the PSLQ algorithm [CITATION NEEDED] to search for
a matching algebraic number or optionally a linear combination
of user-provided base constants (such as $\pi$).

\begin{verbatim}
>>> x = 1 / (sin(pi/5)+sin(2*pi/5)+sin(3*pi/5)+sin(4*pi/5))**2
>>> nsimplify(x)
-2*sqrt(5)/5 + 1
>>> nsimplify(pi, tolerance=0.01)
22/7
>>> nsimplify(1.783919626661888, [pi], tolerance=1e-12)
pi/(-1/3 + 2*pi/3)
\end{verbatim}

\subsection{Polynomials}

\subsection{The Risch Algorithm}
% Also the Meijer-G algorithm, if someone can write about it

\subsection{The Gruntz Algorithm}

\subsection{Logic}

\subsection{Other}


\section{Features}

%% List of Features and how to use
%%
%% Quick overview of the main modules, what it can do and so on. It should probably provide examples how to use sympy.
%%
%% See also the supplement (below)

% Features to discuss in-depth:

% Basic operations (the core)
\subsection{Basic Operations}
% Including the assumptions system

% Calculus (differentiation, integration, limits). Note that algorithm
% descriptions will go in the algorithms section.
\subsection{Calculus}

% Solvers (regular equations, maybe also mention other types of solvers like ODEs/recurrence/Diophantine)
\subsection{Solvers}

% Matrices (worth including to stress that they are symbolic)
\subsection{Matrices}

% Physics module (some sampling, to show that it is there)
\subsection{Physics}

% Series module (Formal Power Series, Fourier Series)
\subsection{Series}

% Series expansion (Differentiate between the two approaches being used)
\subsubsection{Series Expansion}

SymPy has several different implementations for representing series expansions.

\subsubsection{Formal Power Series}

SymPy can be used for computing the Formal Power Series of a function.
The implementation is based on the algorithm described in the paper on Formal Power Series\cite{Gruntz93formalpower}.
The advantage of this approach is that an explicit formula for the coefficients
of the series expansion is generated rather than just computing a few terms.

\begin{verbatim}
>>> from sympy import fps, symbols, sin
>>> x, y = symbols('x')
>>> f = fps(sin(x), x, x0=0)
>>> f.truncate(6)
x - x**3/6 + x**5/120 + O(x**6)
>>> f[15]
-x**15/1307674368000
\end{verbatim}

% Features to list, but not discuss in-depth:

% discrete math, concrete math, plotting, geometry, statistics, polys,
% combinatorics/group theory, code generation, tensors, lie algebras,
% cryptography, category theory, special functions, sets, matrix expressions,
% series, or vectors.
\subsection{Other features}


SymPy includes several packages that allow users to solve domain specific
problems. For example, a comprehensive physics package is included that is
useful for solving problems in classical mechanics, optics, and quantum
mechanics along with support for manipuating physical quantities with units.

\subsection{Vector Algebra}

The \verb|sympy.physics.vector| package provides reference frame, time, and
space aware vector and dyadic objects that allow for three dimensional
operations such as addition, subtraction, scalar multiplication, inner and
outer products, cross products, etc. Both of these objects can be written in
very compact notation that make it easy to express the vectors and dyadics in
terms of multiple reference frames with arbitrarily defined relative
orientations. The vectors are used to specify the positions, velocities, and
accelerations of points, orientations, angular velocities, and angular
accelerations of reference frames, and force and torques. The dyadics are
essentially reference frame aware $3 \times 3$ tensors. The vector and dyadic
objects can be used for any one-, two-, or three-dimensional vector algebra and
they provide a strong framework for building physics and engineering tools.

\begin{listing}
  \begin{minted}{pycon}
>>> from sympy import pi
>>> from sympy.physics.vector import ReferenceFrame
>>> A = ReferenceFrame('A')
>>> B = ReferenceFrame('B')
>>> C = ReferenceFrame('C')
>>> B.orient(A, 'body', (pi, pi / 3, pi / 4), 'zxz')
>>> C.orient(B, 'axis', (pi / 2, B.x))
>>> v = 1 * A.x + 2 * B.z + 3 * C.y
>>> v
A.x + 2*B.z + 3*C.y
>>> v.express(A)
A.x + 5*sqrt(3)/2*A.y + 5/2*A.z
  \end{minted}
  \caption{
    Python interpreter session showing how a vector is created using the
    orthogonal unit vectors of three reference frames that are oriented with
    respect to each other and the result of expressing the vector in the $A$
    frame.
    The $B$ frame is oriented with respect to the $A$ frame using Z-X-Z Euler
    Angles of magnitude $\pi$, $\frac{\pi}{2}$, and
    $\frac{\pi}{3}$\si{\radian},
    respectively whereas the $C$ frame is oriented with respect to the $B$
    frame through a simple rotation about the $B$ frame's X unit vector through
    $\frac{\pi}{2}$\si{\radian}.}
  \label{lis:physics-vector}
\end{listing}

\subsection{Classical Mechanics}

The \verb|physics.mechanics| package utilizes the \verb|physics.vector| package
to populate time aware particle and rigid body objects to fully describe the
kinematics and kinetics of a rigid multi-body system. These objects store all
of the information needed to derive the ordinary differential or differential
algebraic equations that govern the motion of the system, i.e., the equations
of motion. These equations of motion abide by Newton's laws of motion and can
handle any arbitrary kinematical constraints or complex loads. The package
offers two automated methods for formulating the equations of motion based on
Lagrangian Dynamics~\cite{Lagrange1811} and Kane's Method~\cite{Kane1985}. Lastly, there
are automated linearization routines for constrained dynamical
systems based on~\cite{Peterson2014}.

\subsection{Quantum Mechanics}

The \verb|sympy.physics.quantum| package provides quantum functions, states,
operators, and computation of standard quantum models.

% TODO : This needs some help from someone that knows something about quantum
% physics. I wasn't able to understand much from the documentation.

\subsection{Optics}

The \verb|physics.optics| package provides Gaussian optics functions.

% TODO : This needs some help from someone that knows something about optics.

\subsection{Units}

The \verb|physics.units| module provides around two hundred predefined prefixes
and SI units that are commonly used in the sciences. Additionally, it provides
the \verb|Unit| class which allows the user to define their own units.  These
prefixes and units are multiplied by standard SymPy objects to make expressions
unit aware, allowing for algebraic and calculus manipulations to be applied to
the expressions while the units are tracked in the manipulations.  The units of
the expressions can be easily converted to other desired units.  There is also
a new units system in \verb|sympy.physics.unitsystems| that allows the user to
work in specified unit systems.


\section{Other Projects that use SymPy}

\subsection{SymPy Gamma}\label{sympy-gamma}

SymPy Gamma is a simple web application based on Google App Engine that 
executes and displays the results of SymPy expressions as well as
additional related computations, in a fashion similar to that of
Wolfram\textbar{}Alpha. For instance, entering an integer will display
prime factors, digits in the base-10 expansion, and a factorization
diagram. Entering a function will give its docstring; in general,
entering an arbitrary expression will provide the derivative, integral,
series expansion, plot, and roots.

SymPy Gamma also has several additional features than just computing the
results using SymPy.

\begin{itemize}
\item
  It displays integration steps, differentiation steps in detail, which
  can be viewed through this
  \href{http://www.sympygamma.com/input/?i=integrate\%281\%20/\%28\%28x\%2B1\%29\%28x\%2B3\%29\%28x\%2B5\%29\%29\%29}{Link}.
\item
  It also displays the factor tree diagrams for numbers, which is also
  illustrated through this
  \href{http://www.sympygamma.com/input/?i=112}{link}.
\item
  SymPy Gamma also saves user search queries, and offers many such 
  similar features for free, which Wolfram\textbar{}Alpha only offers 
  to it's paid users.
\end{itemize}
Every input query from the user on SymPy Gamma is first, parsed by it's
own parser, which handles several different forms of function names,
which SymPy as a library doesn't support. For instance, SymPy Gamma
supports queries like - \texttt{sin\ x}, whereas SymPy doesn't support
this, and supports only \texttt{sin(x)}.

This parser, then converts the input query to the resultant SymPy
readable code, which is then eventually processed by SymPy and the
result is finally formatted in LaTeX and displayed on the SymPy Gamma
web-app.

\section{Comparison with other CAS}


\subsection{Mathematica}

Wolfram Mathematica is a popular proprietary CAS.
It features highly advanced algorithms.
Mathematica has a core implemented in C++~\cite{Wolfram2016}
which interprets its own programming language (know as Wolfram language).

% M-expressions

Analogously to Lisp's S-expressions,
Mathematica uses its own style of M-expressions,
which are arrays of either atoms or other M-expression.
The first element of the expression identifies the type of the expression
and is indexed by zero, whereas the first argument is indexed by one.
Notice that SymPy expression arguments are stored in a Python tuple
(that is, an immutable array),
while the expression type is identified by the type of the object storing the
expression.

% Attributes

Mathematica can associate attributes to its atoms.

% Expression mutability

Unlike SymPy, Mathematica's expressions are mutable,
that is one can change parts of the expression tree without the need of
creating a new object.
The reactivity of Mathematica allows for a lazy updating of any references
to that data structure.

% * comparison with Mathematica: commutativity, associative expressions, one-identity. Advantage of SymPy: multiplicative commutativity defined on symbols.
% Products and commutativity

Products in Mathematica are determined by some builtin node types,
such as \texttt{Times}, \texttt{Dot}, and others.
\texttt{Times} is overloaded by the * operator,
and is always meant to represent a commutative operator.
The other notable product is \texttt{Dot}, overloaded by the . operator.
This product represents matrix multiplication,
it is not commutative.
SymPy uses the same node for both scalar and matrix multiplication,
the only exception being with abstract matrix symbols.
Unlike Mathematica, SymPy determines commutativity with respect to
multiplication from the factor's expression type.
Mathematica puts the \texttt{Orderless} attribute on the expression
type.

% Associative expressions.

Regarding associative expressions,
SymPy handles associativity by making associative expressions inherit the
class \texttt{AssocOp},
while Mathematica specifies the \texttt{Flat}\cite{WolframRefFlat} attribute on the expression type.

% One identity


% Pattern matching

Mathematica relies heavily on pattern matching:
even the so-called equivalent of function declaration is in reality
the definition of a pattern matching generating an expression tree transformation
on input expressions.
%
Mathematica's pattern matching is sensitive to associative\cite{WolframRefFlat}, commutative\cite{WolframRefOrderless},
and one-identity\cite{WolframRefOneIdentity} properties of its expression tree nodes\cite{WolframRefFlatAndOrderlessFunctions}.
%
SymPy has various ways to perform pattern matching. 
All of them play a lesser role in the CAS than in Mathematica.
looks less mature than Mathematica's,

%% TODO list:
% * comparison with Mathematica: MatrixExp, product not always commutative, type inheritance (polymorphism) and advantage in unifying the product symbol * for symbols and matrices, pattern matching vs. single dispatch.

% Type inheritance and polymorphism

Unlike SymPy, Mathematica does not support type inheritance or polymorphism\cite{Fateman1992}.
% cite examples of class inheritance in SymPy:
% 
SymPy relies heavily on class inheritance, but for the most part,
class inheritance is used to make sure that SymPy objects inherit the proper
methods and implement the basic hashing system.
Associativity of expressions can be achieved by inheriting the class \texttt{AssocOp},
which may appear a more cumbersome operation than Mathematica's attribute setting.
%There are also cases where inheritance is used to extend the mathematical meaning of an expression.

% Matrices

Matrices in SymPy are types on their own.
In Mathematica, nested lists are interpreted as matrices whenever the sublists
have the same length.
The main difference to SymPy is that ordinary operators and functions 
do not get generalized the same way as used in traditional mathematics.
Using the standard multiplication in Mathematica performs an elementwise
product, this is compatible with Mathematica's convention of commutativity of
\texttt{Times} nodes. 
Matrix product is expressed by the \textit{dot} operator, 
or the \texttt{Dot} node.
The same is true for the other operators, and even functions,
most notably calling the exponential function \texttt{Exp} on a matrix 
returns an elementwise exponentiation of its elements.
The real matrix exponentiationl is available through the \texttt{MatrixExp}
function.

% * comparison with Mathematica: avoid misspelling variables through forced declaration (check that you can't do it in Mathematica).
% * evaluate=False vs HoldForm

Unevaluated expressions can be achieved in various ways,
most commonly with the \texttt{HoldForm} or \texttt{Hold} nodes,
that block the evaluation of subnodes by the parser.
Note that such a node cannot be expressed in Python, because of greedy evaluation.
Whenever needed in SymPy, it is necessary to add the parameter \texttt{evaluate=False}
to all subnodes, or put the input expression in a string.

% * comparison with Mathematica: == is structural equality, not 

The operator == returns a boolean whenever it is able to immediately evaluate
the truthness of the equality, otherwise it returns an \texttt{Equal} expression.
In SymPy == means structural equality and is always guaranteed to return a
boolean expression.
To express an equality in SymPy it is necessary to explicitly construct the 
\texttt{Equality} class.

% * comparison with Mathematica: polynomial module.
% * comparison with Mathematica: space is product, ** vs ^

SymPy, in accordance with Python and unlike the usual programming convention,
uses ** to express the power operator, while Mathematica uses the more 
common ^.

% * comparison with Mathematica: ( ) is Sequence, functions are generally uppercase.
% * comparsion with Mathematica: table of comparison?
% * comparison with Mathematica: Wolfram language has loads of operator overloading, functional paradigm.



\section{Conclusion and future work}

SymPy provides a wide array of features for symbolic algebra. It is designed
to be used in an extensible way, both as a end-user application and as a
library. It

The goal of SymPy is somewhat lofty: to be a full-featured CAS\@. However, a
few things sit at the head of the project's roadmap.

% Performance. Mention SymEngine.

%


\section{References}

\bibliography{paper}
\bibliographystyle{unsrt}

\end{document}
