% Series expansion (Differentiate between the two approaches being used)
\subsubsection{Series Expansion}

SymPy is able to calculate the symbolic series expansion of an arbitrary series
or expression involving elementary and special functions and multiple
variables. For this it has two different implementations- the \texttt{series}
method and Ring Series.

The first approach stores a series as an object of the \texttt{Basic} class.
Each function has its specific implementation of its expansion which is able to
evaluate the Puiseux series expansion about a specified point. For example,
consider a Taylor expansion about 0:

\begin{verbatim}
>>> from sympy import symbols, series
>>> x, y = symbols('x, y')
>>> series(sin(x+y) + cos(x*y), x, 0, 2)
1 + sin(y) + x*cos(y) + O(x**2)
\end{verbatim}

The newer and much faster\cite{sympyRingSeries} approach called Ring Series makes use of the
observation that a truncated Taylor series, is in fact a polynomial.
Ring Series uses the efficient representation and operations of sparse
polynomials. The choice of sparse polynomials is deliberate as it performs
well in a wider range of cases than a dense representation. Ring Series gives
the user the freedom to choose the type of coefficients he wants to have in
his series, allowing the use of faster operations on certain types.

For this, several low level methods for expansion of trigonometric, hyperbolic
and other elementary functions like inverse of a series, calculating $n$th
root, etc, are implemented using variants of the Newton\cite{zimmerman} Method.
All these support Puiseux series expansion. The following example demonstrates
the use of an elementary function that calculates the Taylor expansion of the
\texttt{sine} of a series.

\begin{verbatim}
>>> from sympy import ring
>>> from sympy.polys.ring_series import rs_sin
>>> R, t = ring('t', QQ)
>>> rs_sin(t**2 + t, t, 5)
-1/2*t**4 - 1/6*t**3 + t**2 + t
\end{verbatim}

The function \texttt{sympy.polys.rs\_series} makes use of these elementary
functions to expand an arbitrary SymPy expression. It does so by following a
recursive strategy of expanding the lower most functions first and then
composing them recursively to calculate the desired expansion. Currently it
only supports expansion about 0 and is under active development. Ring Series
is several times faster than the default implementation with the speed
difference increasing with the size of the series. The
\texttt{sympy.polys.rs\_series} takes as input any SymPy expression and hence
there is no need to explicitly create a polynomial \texttt{ring}. An example:

% no-doctest
\begin{verbatim}
>>> from sympy.polys.ring_series import rs_series
>>> from sympy.abc import a, b
>>> from sympy import sin, cos
>>> rs_series(sin(a + b), a, 4)
-1/2*(sin(b))*a**2 + (sin(b)) - 1/6*a**3*(cos(b)) + a*(cos(b))
\end{verbatim}

\subsubsection{Formal Power Series}

SymPy can be used for computing the Formal Power Series of a function.
The implementation is based on the algorithm described in the paper on
Formal Power Series\cite{Gruntz93formalpower}.  The advantage of this approach is
that an explicit formula for the coefficients of the series expansion is generated
rather than just computing a few terms.

The following example shows how to use \texttt{fps}:

\begin{verbatim}
>>> f = fps(sin(x), x, x0=0)
>>> f.truncate(6)
x - x**3/6 + x**5/120 + O(x**6)
>>> f[15]
-x**15/1307674368000
\end{verbatim}

\subsubsection{Fourier Series}

SymPy provides functionality to compute Fourier Series of a function using
the \texttt{fourier\_series} function. Under the hood it just computes $a0$, $an$, $bn$ using
standard integration formulas.

Here's an example on how to compute Fourier Series in SymPy:

\begin{verbatim}
>>> L = symbols('L')
>>> f = fourier_series(2 * (Heaviside(x/L) - Heaviside(x/L - 1)) - 1, (x, 0, 2*L))
>>> f.truncate(3)
4*sin(pi*x/L)/pi + 4*sin(3*pi*x/L)/(3*pi) + 4*sin(5*pi*x/L)/(5*pi)
\end{verbatim}
